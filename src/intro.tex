\chapter*{INTRODUCCIÓN}
\addcontentsline{toc}{chapter}{INTRODUCCIÓN}

% \mybox{Introducción al trabajo de tesis}

El proceso de formado de lámina en frío mediante operaciones de formado es uno de los métodos más 
utilizados para dar forma a los metales con espesor relativamente delgado. Normalmente estas láminas 
tienen la ventaja de poseer altos módulos de elasticidad y valores de esfuerzo de fluencia aceptables 
lo que proporciona una buena rigidez y una excelente relación de resistencia/peso. Dichas cualidades 
han propiciado una creciente necesidad en la industria para el desarrollo, en los últimos años, 
de mejores técnicas y procesos de manufactura en busca de optimizar la calidad de sus productos y al 
mismo tiempo disminuir los costos productivos, basados en estudios de la teoría de la plasticidad y la 
capacidad de formado de los metales.\\

A la par, las técnicas de análisis de los procesos de formado por computadora se han venido 
perfeccionando para alcanzar resultados muy precisos en tiempo más reducidos; día con día las 
empresas del sector han incorporado estas herramientas para los procesos de desarrollo, validación, 
y modificación de troqueles de embutido.\\

En este trabajo de tesis se presenta la simulación por elementos finitos de un proceso de formado 
de un tubo de acero AISI 1018 que se utiliza como parte de un buje para suspensiones automotrices, 
utilizando un análisis de tipo dinámico explícito que permite simular este tipo de procesos 
de una manera bastante aceptable, considerando las diversas complejidades que implican este tipo 
de análisis.\\

Este texto se divide en cuatro capítulos principales a saber: marco de referencia, marco teórico, 
metodología utilizada y el análisis de resultados, además de las conclusiones y los anexos respectivos.\\

En el capítulo 1 correspondiente al marco teórico, se detallan los antecedentes, la descripción 
del problema abordado, los objetivos generales y específicos, las limitaciones, y una 
revisión del estado del arte referente a la simulación de procesos de estampado utilizando 
el método de los elementos finitos.\\

En el capítulo 2 se presenta la información referente al fundamento teórico necesario para el 
desarrollo del proyecto. El contenido está divido en cuatro secciones. La primera referente 
a los procesos de formado, haciendo énfasis en las operaciones de doblado y la 
característica general de los herramentales utilizados. La segunda sección aborda 
los conceptos fundamentales de la mecánica de sólidos, tales como deformaciones, esfuerzos, 
criterios de fluencia. La tercera sección es una introducción al método de los elementos finitos, 
el proceso de discretización, la teoría de contactos, el enfoque implícito y explícito en 
la solución de problemas de elementos finitos, así como algunas consideraciones generales 
sobre la aplicación del método en problemas de ingenería. La sección \ref{sec:extensometria} 
contiene información acerca de la técnica de extensometría aplicada al análisis experimental 
de esfuerzos y deformaciones. \\

En el capítulo 3 se describe la metodología seguida para el desarrollo del proyecto, 
desde la documentación del proyecto, la caracterización del material, el análisis por 
elemento finito, el análisis experimental, y cada una de las etapas necesarias para 
la conclusión y consecución de los objetivos planteados.\\

En el capítulo 4 se presenta el análisis y discusión de los principales resultados 
obtenidos en el desarrollo de este proyecto. Del análisis por elementos finitos 
se presentan los resultados del caso bidimensional y del modelo tridimensional, comparando 
ambos casos, con la finalidad de discutir la viabilidad de utilizar un análisis de tipo 
deformación plana en una simulación de formado como esta. Además los resultados del 
análisis numérico se comparan con lo obtenido mediante el análisis experimental.
Con la finalidad de determinar algunas condiciones o parámetros que faciliten 
el análisis de procesos de formado de este tipo, se presenta una sección destinada 
a evaluar la influencia del escalamiento de masa selectivo cuya utilidad va en 
dirección de disminuir el tiempo de cómputo requerido. Asimismo se expone una sección 
referente al uso de algunos modelos de material como el bilineal isotrópico y cinemático, 
la plasticidad cinemática y curvas multilineales, para determinar si las variaciones 
en el comportamiento del material simulado son significativas.\\

