\chapter*{Resumen}
\addcontentsline{toc}{chapter}{Resumen}

% \mybox{Resumen del trabajo de tesis. No mayor a 20 líneas.}

La simulación por elementos finitos de procesos de formado es una herramienta auxiliar en el proceso de 
diseño y optimización de herramentales. En este trabajo de tesis se presenta el análisis del proceso de formado 
de un tubo de acero AISI 1018 mediante un troquel compuesto de dos etapas, a saber: un doblado en U y un 
cerrado de tubo.\\

Se utilizó un análisis de tipo dinámico explícito realizado mediante el software de simulación 
ANSYS/LS-DYNA\CR, tomando en cuenta todas las no linealidades inherentes al proceso real, tales 
como el modelo del material, las geometrías, la naturaleza dinámica del fenónemo y la interacción 
entre los diversos componentes del herramental y la pieza de trabajo.\\

Con la finalidad de evaluar el comportamiento de la pieza de trabajo durante el proceso de trabajo 
se utilizaron cuatro modelos de material: curva multilineal, bilineal isotrópico, bilineal cinemático 
y plasticidad cinemática, obteniéndose resultados muy similares para la variación del esfuerzo de 
von Mises en todos los casos, y con diferencias menores a un 20\% en el caso de la fuerza de formado requerida. \\

Se realizó un análisis comparativo para diversos casos de escalamiento de masa selectiva, con el objetivo 
de determinar su influencia en los resultados obtenidos. De lo anterior se calculó que agregando alrededor de 
un 110\% de \textit{masa artificial} es posible reducir en un 90\% el tiempo de cómputo, con variaciones mínimas 
de los resultados de fuerza y esfuerzo de von Mises obtenidos.\\
