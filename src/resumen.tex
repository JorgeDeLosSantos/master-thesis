\addcontentsline{toc}{chapter}{Resumen}
\chapter*{Resumen}

% \mybox{Resumen del trabajo de tesis. No mayor a 20 líneas.}

La simulación por elemento finito de procesos de estampado es una herramienta auxiliar en el 
proceso de diseño y optimización de herramentales. En este artículo se presenta el análisis 
del proceso de formado de un tubo de acero SAE 1018 mediante un troquel progresivo de dos etapas. 
Se utilizó ANSYS/LS-DYNA para realizar las simulaciones de tipo dinámico explícito, considerando 
las diversas no linealidades que implican este tipo de análisis, derivadas de las propiedades 
del material, las grandes deformaciones y los contactos entre las herramientas y la pieza de trabajo. 