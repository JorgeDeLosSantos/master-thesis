\addcontentsline{toc}{chapter}{Resumen}
\chapter*{Resumen}

% \mybox{Resumen del trabajo de tesis. No mayor a 20 líneas.}

La simulación por elemento finito de procesos de estampado es una herramienta auxiliar en el proceso de diseño y optimización de herramentales. En este artículo se presenta el análisis del proceso de formado de un tubo de acero AISI 1018 mediante un troquel progresivo de dos etapas. Se utilizó ANSYS/LS-DYNA para realizar las simulaciones de tipo dinámico explícito, considerando las diversas no linealidades que implican este tipo de análisis, derivadas de las propiedades del material, las grandes deformaciones y los contactos entre las herramientas y la pieza de trabajo. Los resultados obtenidos son la fuerza máxima requerida para el proceso de formado inicial que fue de 1.27 toneladas y los esfuerzos máximos que se generan en el proceso de doblez que fueron de 76 213 Psi. Se muestra, además, la geometría resultante del proceso de formado comparada de forma cualitativa con la obtenida mediante la simulación.