\chapter{Marco teórico}

\section{Mecánica de sólidos}

\subsection{Relaciones constitutivas}

En el caso de un sólido tridimensional isotrópico la relación esfuerzo-deformación está dada por la Ley de Hooke, 
expresada en términos matemáticos como:

\begin{equation}\label{eq:ecdef}
\vec{\varepsilon} = 
\left\{\begin{matrix}
\varepsilon_{xx} \\ \varepsilon_{yy} \\ \varepsilon_{zz} \\ \varepsilon_{xy} \\ \varepsilon_{yz} \\ \varepsilon_{zx}
\end{matrix}\right\} = 
\left[ C \right] \vec{\sigma} + \vec{\varepsilon_0} = 
\left[ C \right]
\left\{\begin{matrix}
\sigma_{xx} \\ \sigma_{yy} \\ \sigma_{zz} \\ \sigma_{xy} \\ \sigma_{yz} \\ \sigma_{zx}
\end{matrix}\right\} + 
\left\{\begin{matrix}
\varepsilon_{xx}_0 \\ \varepsilon_{yy}_0 \\ \varepsilon_{zz}_0 \\ \varepsilon_{xy}_0 \\ \varepsilon_{yz}_0 \\ \varepsilon_{zx}_0
\end{matrix}\right\}
\end{equation}

Donde $[C]$ es la matriz de constantes elásticas definida por:

\begin{equation}
[C] = \frac{1}{E}
\left[\begin{matrix}
1 & -\nu & -\nu & 0 & 0 & 0 \\
-\nu & 1 & -\nu & 0 & 0 & 0 \\
-\nu & -\nu & 1 & 0 & 0 & 0 \\
0 & 0 & 0 & 2(1+\nu) & 0 & 0 \\
0 & 0 & 0 & 0 & 2(1+\nu) & 0 \\
0 & 0 & 0 & 0 & 0 & 2(1+\nu) \\
\end{matrix}\right]
\end{equation}

$\vec{\varepsilon_0}$ es el vector de esfuerzos iniciales, $E$ es el módulo de Young y $\nu$ el coeficiente de Poisson del 
material.\\

Algunas veces la expresión de esfuerzos en términos de deformaciones puede ser necesaria. Incluyendo las deformaciones 
térmicas, la ecuación \ref{eq:ecdef} puede ser invertida para obtener:

\begin{equation}
\vec{\sigma} = 
\left\{\begin{matrix}
\sigma_{xx} \\ \sigma_{yy} \\ \sigma_{zz} \\ \sigma_{xy} \\ \sigma_{yz} \\ \sigma_{zx}
\end{matrix}\right\} = 
\left[ D \right] (\vec{\varepsilon} - \vec{\varepsilon_0})= 
\left[ D \right] 
\left\{\begin{matrix}
\varepsilon_{xx} \\ \varepsilon_{yy} \\ \varepsilon_{zz} \\ \varepsilon_{xy} \\ \varepsilon_{yz} \\ \varepsilon_{zx}
\end{matrix}\right\} - 
\frac{E\alpha T}{1-2\nu}
\left\{\begin{matrix}
1 \\ 1 \\ 1 \\ 0 \\ 0 \\ 0
\end{matrix}\right\} 
\end{equation}

Donde la matriz $[D]$ está dada por:

\begin{equation}
[C] = \frac{E}{(1+\nu)(1-2\nu)}
\left[\begin{matrix}
1-\nu & \nu & \nu & 0 & 0 & 0 \\
\nu & 1-\nu & \nu & 0 & 0 & 0 \\
\nu & \nu & 1-\nu & 0 & 0 & 0 \\
0 & 0 & 0 & \frac{1-2\nu}{2} & 0 & 0 \\
0 & 0 & 0 & 0 & \frac{1-2\nu}{2} & 0 \\
0 & 0 & 0 & 0 & 0 & \frac{1-2\nu}{2} \\
\end{matrix}\right]
\end{equation}

En el caso de problemas bidimensionales, dos tipos de distribuciones de esfuerzos son posibles: esfuerzo plano 
y deformación plana.

\subsubsection{Esfuerzo plano}

La consideración de esfuerzo plano es aplicable para cuerpos en los cuales su dimensión en una dirección es muy 
pequeña comparada con las otras. En el caso de esfuerzo plano se asume que:

\begin{equation}
\sigma_{zz} = \sigma_{zx} = \sigma_{yz} = 0
\end{equation}

Donde $z$ representa la dirección perpendicular al plano de la placa mostrada en la figura \ref{fig:fg1}, 
y los componentes de esfuerzo no varían a través del espesor de la placa.

\begin{center}
\includegraphics[scale=0.35]{src/plane_stress.png}
\captionof{figure}{Esfuerzo plano}
\label{fig:fg1}
\end{center}

Entonces, las relaciones esfuerzo deformación se reducen a:

\begin{equation}
\vec{\varepsilon} = [C] \vec{\sigma} + \vec{\varepsilon_0}
\end{equation}

donde:

\begin{equation}
\vec{\varepsilon} = 
\left\{\begin{matrix}
\varepsilon_{xx} \\ \varepsilon_{yy} \\ \varepsilon_{xy}
\end{matrix}\right\}
\end{equation}

\begin{equation}
\vec{\sigma} = 
\left\{\begin{matrix}
\sigma_{xx} \\ \sigma_{yy} \\ \sigma_{xy}
\end{matrix}\right\}
\end{equation}

\begin{equation}
[C] = \frac{1}{E}
\left[\begin{matrix}
1 & -\nu & 0 \\
-\nu & 1 & 0 \\
0 & 0 & 2(1+\nu) \\
\end{matrix}\right]
\end{equation}

\begin{equation}
\vec{\varepsilon_0} = 
\left\{\begin{matrix}
\varepsilon_{xx_0} \\ \varepsilon_{yy_0} \\ \varepsilon_{xy_0}
\end{matrix}\right\} =
\alpha T 
\left\{\begin{matrix}
1 \\ 1 \\ 0 \\
\end{matrix}\right\}
\end{equation}

En el caso de deformaciones térmicas,

\begin{equation}
\vec{\sigma} = [D] (\vec{\varepsilon} - \vec{\varepsilon_0}) = 
[D] \vec{\varepsilon} - 
\frac{E\alpha T}{1-\nu} 
\left\{\begin{matrix}
1 \\ 1 \\ 0 \\
\end{matrix}\right\}
\end{equation}

Con, 

\begin{equation}
[D] = \frac{E}{1-\nu^2}
\left[\begin{matrix}
1 & \nu & 0 \\
\nu & 1 & 0 \\
0 & 0 & \frac{1-\nu}{2} \\
\end{matrix}\right]
\end{equation}

En el caso de esfuerzo plano la componente de la deformación en el plano $z$, será diferente de cero, debido al 
efecto del coeficiente de Poisson, estando dada  por la expresión:

\begin{equation}
\varepsilon_{zz} = -\frac{\nu}{E} (\sigma_{xx} + \sigma_{yy}) + \alpha T = 
\frac{-\nu}{1-\nu} (\varepsilon_{xx} + \varepsilon_{yy}) +  
\frac{1+\nu}{1-\nu} \alpha T
\end{equation}

Mientras que, 

\begin{equation}
\varepsilon_{yz} = \varepsilon_{zx} = 0
\end{equation}


\subsubsection{Deformación plana}

La consideración de deformación plana es aplicable para sólidos largos y cuya geometría y cargas no varían de 
manera significativa en la dirección longitudinal.\\

En este caso las ecuaciones de esfuerzo-deformación tridimensional se reducen a:

\begin{equation}
\vec{\varepsilon} = [C] \vec{\sigma} + \vec{\varepsilon_0}
\end{equation}

donde:

\begin{equation}
\vec{\varepsilon} = 
\left\{\begin{matrix}
\varepsilon_{xx} \\ \varepsilon_{yy} \\ \varepsilon_{xy}
\end{matrix}\right\}
\end{equation}

\begin{equation}
\vec{\sigma} = 
\left\{\begin{matrix}
\sigma_{xx} \\ \sigma_{yy} \\ \sigma_{xy}
\end{matrix}\right\}
\end{equation}

\begin{equation}
[C] = \frac{1+\nu}{E}
\left[\begin{matrix}
1-\nu & -\nu & 0 \\
-\nu & 1-\nu & 0 \\
0 & 0 & 2 \\
\end{matrix}\right]
\end{equation}

\begin{equation}
\vec{\varepsilon_0} = 
\left\{\begin{matrix}
\varepsilon_{xx_0} \\ \varepsilon_{yy_0} \\ \varepsilon_{xy_0}
\end{matrix}\right\} =
(1+\nu) \alpha T 
\left\{\begin{matrix}
1 \\ 1 \\ 0 \\
\end{matrix}\right\}
\end{equation}

En el caso de deformaciones térmicas,

\begin{equation}
\vec{\sigma} = [D] (\vec{\varepsilon} - \vec{\varepsilon_0}) = 
[D] \vec{\varepsilon} - 
\frac{E\alpha T}{1-2\nu} 
\left\{\begin{matrix}
1 \\ 1 \\ 0 \\
\end{matrix}\right\}
\end{equation}

Con, 

\begin{equation}
[D] = \frac{E}{(1+\nu)(1-2\nu)}
\left[\begin{matrix}
1-\nu & \nu & 0 \\
\nu & 1-\nu & 0 \\
0 & 0 & \frac{1-2\nu}{2} \\
\end{matrix}\right]
\end{equation}

\begin{center}
\includegraphics[scale=0.55]{src/plane_strain.png}
\captionof{figure}{Deformación plana}
\label{fig:fg2}
\end{center}

El componente de esfuerzo en la dirección $z$ no será nulo debido a los efectos del coeficiente de Poisson, y está 
dado por:

\begin{equation}
\sigma_{zz} = \nu(\sigma_{xx}+\sigma_{yy}) - E \alpha T
\end{equation}

y, 

\begin{equation}
\sigma_{yz} = \sigma_{zx} = 0
\end{equation}

\subsection{Relaciones deformación-desplazamiento}

La forma deformada de un cuerpo elástico bajo una determinada configuración de cargas y distribución de temperaturas 
pueden ser descritas completamente por tres componentes de desplazamiento $u, v$ y $w$ paralelas a las direcciones 
$x, y$ y $z$ respectivamente. En general, cada una de estas componentes $u,v$ y $w$ es una función de las 
coordenadas $x,y$ y $z$. Las deformaciones inducidas en el sólido pueden ser expresadas en términos de los 
desplazamientos $u,v$ y $w$.\\

Si los desplazamientos se consideran muy pequeños, la deformaciones pueden ser expresadas como:

\begin{subequations}
\begin{eqnarray}
\varepsilon_{xx} = \frac{du}{dx} \\
\varepsilon_{yy} = \frac{dv}{dy} \\
\varepsilon_{zz} = \frac{dw}{dz} \\
\varepsilon_{xy} = \frac{du}{dy} + \frac{dv}{dx} \\
\varepsilon_{yz} = \frac{dw}{dy} + \frac{dv}{dz} \\
\varepsilon_{zx} = \frac{du}{dz} + \frac{dw}{dx} \\
\end{eqnarray}
\end{subequations}

\subsection{Plasticidad}

La teoría de plasticidad estudia la fluencia de materiales bajo estados de esfuerzos complejos. Permite 
conocer si un material cederá bajo ciertas condiciones de esfuerzo y determinar el cambio en la forma o 
geometría en caso de que la fluencia ocurra. También permite usar datos de ensayos de tensión para predecir 
el endurecimiento por carga durante la deformación bajo complejos estados de esfuerzo. Estas relaciones 
son parte fundamental de los códigos de computadora utilizados para predecir la capacidad de una estructura 
para absorber impactos, así como en procesos de formado o estampado que involucran la deformación plástica de 
placas metálicas.

\subsubsection{Criterio de fluencia}

Un criterio de fluencia es una expresión matemática propuesta del estado de esfuerzo que causará la fluencia. La forma más general es:

\begin{equation}
f(\sigma_x,\sigma_y, \sigma_z, \tau_{yz}, \tau_{zx}, \tau_{xy} ) = C 
\end{equation}

Para materiales isotrópicos esto puede ser expresado en términos de los esfuerzos principales como:

\begin{equation}
f(\sigma_1,\sigma_2,\sigma_3 )=C
\end{equation}

Para la mayoría de los metales dúctiles isotrópicos comúnmente se hacen las siguientes consideraciones:

\begin{itemize}
\item El esfuerzo de fluencia en tensión y compresión es el mismo.
\item El volumen permanece constante durante la deformación plástica
\item La magnitud del esfuerzo normal promedio, no afecta la fluencia.
\end{itemize}

\begin{equation}
\sigma_m=\frac{\sigma_1+\sigma_2+\sigma_3}{3}
\end{equation}

La consideración que la fluencia es independiente de $\sigma_m$ es razonable porque la deformación usualmente ocurre por deslizamiento o mecanismos de corte. Por lo tanto los criterios de fluencia para materiales isotrópicos tienen la forma:

\begin{equation}
f[(\sigma_2-\sigma_3 ),(\sigma_3-\sigma_1 ),(\sigma_1-\sigma_2 )] = C
\end{equation}

\subsubsection{Criterio de Tresca}

El criterio más simple es uno de los primero propuestos por Tresca. Afirma que la cedencia ocurrirá cuando el máximo 
esfuerzo cortante alcance un valor crítico. El máximo esfuerzo cortante viene dado por:

\begin{equation}
\tau_{max} = \frac{\sigma_{max}-\sigma_{min}}{2}
\end{equation}

entonces, el criterio de Tresca puede expresarse como:

\begin{equation}
\sigma_{max} - \sigma_{min} = C
\end{equation}

Si se mantiene la convención de que $ \sigma_1 \me \sigma_2 \me \sigma_3 $, puede reescribirse lo anterior como:

\begin{equation}\label{eq:ec1}
\sigma_1 - \sigma_3 = C
\end{equation}

La constante C puede ser encontrada mediante un ensayo de tensión uniaxial. En un ensayo de tensión, 
$\sigma_2 = \sigma_3 = 0$ y la cedencia $\sigma_1 = Y$, donde $Y$ es el esfuerzo de fluencia. Sustituyendo 
en $C=Y$ en la ecuación \ref{eq:ec1}. Por lo tanto el criterio de Tresca puede ser expresado como:

\begin{equation}\label{eq:ec2}
\sigma_1 - \sigma_3 = Y
\end{equation}

Para cortante puro, $ \sigma_1 = -\sigma_3 = k$, donde $k$ es esfuerzo de fluencia por cortante. Sustituyendo 
$ k = Y/2 $ en la ecuación \ref{eq:ec2}, entonces:

\begin{equation}
\sigma_1 - \sigma_3 = 2k = C
\end{equation}

\subsubsection{Criterio de Von Mises}

El efecto del esfuerzo principal medio puede ser incluido asumiendo que la fluencia depende de la raíz cuadrada 
del promedio de los diámetros de los tres círculos de Mohr. Este es el criterio de Von Mises, el cual puede ser 
expresado como:

\begin{equation} \label{eq:ec3}
\left\{ [(\sigma_2-\sigma_3)^2 + (\sigma_3-\sigma_1 )^2 + (\sigma_1-\sigma_2 )^2]/3 \right\}^{1/2} = C
\end{equation}

Como cada término está elevado al cuadrado, la convención $\sigma_1 \me \sigma_2 \me \sigma_3$ no es necesaria. 
La constante del material, $C$, puede ser evaluada mediante un ensayo de tensión uniaxial. En la fluencia, 
$\sigma_1 = Y$ y $\sigma_2 = \sigma_3 = 0$. Sustituyendo, $[0^2 + (-Y)^2 + Y^2]/3 = C^2$, o 
$C = (2/3)^{1/3} Y $, entonces, la ecuación usualmente se escribe como:

\begin{equation}\label{eq:ec4}
(\sigma_2-\sigma_3)^2 + (\sigma_3-\sigma_1 )^2 + (\sigma_1-\sigma_2 )^2 = 2Y^2
\end{equation}

Para cortante puro, $\sigma_1 = -\sigma_3 = k$ y $\sigma_2=0$. Sustituyendo en la ecuación \ref{eq:ec4}, 
$ (-k)^2 + [ (-k)-k ]^2 + k^2 = 2Y^2 $, entonces:

\begin{equation}\label{eq:ec5}
k = Y/\sqrt{3}
\end{equation}

La ecuación \ref{eq:ec5} puede ser simplificada si uno de los esfuerzos principales es cero (condición de esfuerzo plano). 
Sustituyendo $\sigma_3 = 0$, $\sigma_1^2 + \sigma_2^2 - \sigma_1 \sigma_2 = Y^2$, el cual es una elipse. Con la 
consiguiente sustitución de $\alpha = \sigma_2/\sigma_1$,

\begin{equation}
\sigma_1 = Y/(1-\alpha+\alpha^2)^{1/2}
\end{equation}


\subsection{Ecuaciones de equilibrio}

\subsubsection{Equilibrio externo}

Si un sólido está en equilibrio bajo ciertas condiciones de carga, las fuerzas reactivas y momentos desarrollados en 
los puntos de soporte deben balancear las fuerzas y momentos externos aplicados. Haciendo referencia a la figura 
\ref{fig:externalforces}, sean $\phi_x, \phi_y, \phi_z$ las fuerzas de cuerpo, $\Phi_x, \Phi_y, \Phi_z$ las 
fuerzas de superficie, $P_x, P_y, P_z$ las fuerzas concentradas externas, y $Q_x, Q_y, Q_z$ los momentos externos 
aplicados. Entonces, las ecuaciones de equilibrio externo pueden establecerse como:

\begin{eqnarray}
\int_S \Phi_x ds + \int_V \phi_x dV + \sum P_x = 0 \nonumber \\
\int_S \Phi_y ds + \int_V \phi_y dV + \sum P_y = 0 \\
\int_S \Phi_z ds + \int_V \phi_z dV + \sum P_z = 0 \nonumber
\end{eqnarray}

\begin{center}
\includegraphics[scale]{src/equillibrium_force.png}
\captionof{figure}{Fuerzas de equilibrio externo}
\label{fig:externalforces}
\end{center}

Para el equilibrio de momentos:

\begin{eqnarray}
\int_S (\Phi_zy - \Phi_{y}z) ds + \int_V (\phi_zy - \phi_{y}z) dV + \sum Q_x = 0 \nonumber \\
\int_S (\Phi_x z - \Phi_z x) ds + \int_V (\phi_x z - \phi_z x) dV + \sum Q_y = 0 \\
\int_S (\Phi_{y}x - \Phi_x y) ds + \int_V (\phi_{y}x - \phi_x y) dV + \sum Q_z = 0 \nonumber
\end{eqnarray}

Donde $S$ es la superficie y $V$ el volumen del cuerpo sólido.


\subsubsection{Equilibrio interno}

Debido a las aplicación de cargas, se desarrollan esfuerzos dentro del sólido. Si consideramos 
un elemento de material dentro del sólido, este debería estar en equilibrio debido a esos 
esfuerzos internos desarrollados.\\

Teoricamente, el estado de esfuerzos en cualquier punto de un cuerpo cargado está completamente definido 
en términos de nueve componentes de esfuerzo: $\sigma_{xx}, \sigma_{yy}, \sigma_{zz}, \sigma_{xy}, \sigma_{yx}, 
\sigma_{yz}, \sigma_{zy}, \sigma_{zx}, \sigma_{xz} $, donde los primeros tres son las componentes normales y los restantes 
son esfuerzos cortantes. Las ecuaciones de equilibrio interno relativas a los nueve componentes de esfuerzo pueden ser 
obtenidas considerando el equilibrio de momentos y fuerzas actuando en el volumen elemental mostrado en 
la figura \ref{fig:internalforces}. El equilibrio de momentos alrededor de los ejes $x,y,z$, asumiendo que no existen 
momentos en el sólido, derivan en las siguientes relaciones:

\begin{equation}
\sigma_{yx} = \sigma_{xy}, \,\,\,\, \sigma_{zy} = \sigma_{yz}, \,\,\,\, \sigma_{xz} = \sigma_{zx}
\end{equation}

\begin{center}
\includegraphics[scale]{src/internal_force.png}
\captionof{figure}{Fuerzas de equilibrio interno}
\label{fig:internalforces}
\end{center}

Estas ecuaciones muestran que el estado de esfuerzos en cualquier punto puede ser completamente definido por 
los seis componentes $\sigma_{xx}, \sigma_{yy}, \sigma_{zz}, \sigma_{xy}, \sigma_{yz}, \sigma_{zx} $. 
El equilibrio de fuerzas en las direcciones $x,y,z$ proporcionan las siguientes ecuaciones diferenciales de 
equilibrio:

\begin{eqnarray}
\frac{\partial \sigma_{xx}}{\partial x} + \frac{\partial \sigma_{xy}}{\partial y} + 
\frac{\partial \sigma_{zx}}{\partial z} + \phi_{x} = 0 \nonumber \\
\frac{\partial \sigma_{xy}}{\partial x} + \frac{\partial \sigma_{yy}}{\partial y} + 
\frac{\partial \sigma_{yz}}{\partial z} + \phi_{y} = 0 \\
\frac{\partial \sigma_{zx}}{\partial x} + \frac{\partial \sigma_{yz}}{\partial y} + 
\frac{\partial \sigma_{zz}}{\partial z} + \phi_{z} = 0 \nonumber
\end{eqnarray}

donde $\phi_x, \phi_y$ y $\phi_z$, son las fuerzas de cuerpo por unidad de volumen actuando a lo largo 
de las direcciones $x,y$ y $z$, respectivamente.\\

Para un problema bidimensional, existen solamente tres componentes de esfuerzo independientes, 
$(\sigma_{xx},\sigma_{yy},\sigma_{xy})$ y las ecuaciones de equilibrio se reducen a:

\begin{eqnarray}
\frac{\partial \sigma_{xx}}{\partial x} + \frac{\partial \sigma_{xy}}{\partial y} + \phi_x = 0 \nonumber \\
\frac{\partial \sigma_{xy}}{\partial x} + \frac{\partial \sigma_{yy}}{\partial y} + \phi_y = 0 
\end{eqnarray}

En problemas unidimensionales, sólo estará presente un componente de esfuerzo: $\sigma_{xx}$. Entonces, 
las ecuaciones de equilibrio se reducen a:

\begin{equation}
\frac{\partial \sigma_{xx}}{\partial x} + \phi_x = 0
\end{equation}


\section{El método de los elementos finitos}

\subsection{Generalidades del método}

En el método de elementos finitos se considera un cuerpo continuo o sólido, como un ensamble de pequeñas subdivisiones llamadas elementos finitos. Estos elementos están interconectados a través de nodos comunes. Debido a que la variación real de las variables de campo (desplazamientos, esfuerzos, temperaturas, etc.) se desconoce en el continuo, se asume que la variación de estas en el modelo de elemento finito puede ser aproximada por una simple función. Estas funciones de aproximación, también llamadas modelos de interpolación, son definidas en términos de los valores nodales de las variables de campo.\\

En general, el método de los elementos finitos, consiste en formular un sistema de ecuaciones (ecuaciones de equilibrio) para el sistema continuo que ha sido discretizado, donde las incógnitas suelen ser los valores nodales de las variables de campo. Luego, se resuelve este sistema de ecuaciones, con las consideraciones correspondientes a las condiciones de frontera o valores iniciales que simplifiquen el modelo original. La siguiente ecuación muestra, en notación matricial, el sistema de ecuaciones resuelto en una formulación de elemento finito.

\begin{equation}
\bm{\left[K\right] \vec{u} = \vec{P}}
\end{equation}

Donde $\bm{[K]}$ es la matriz global de rigidez, $\bm{\vec{u}}$ es el vector de desplazamientos nodales y $\bm{\vec{P}}$ el vector de fuerzas nodales en el sistema.\\

Para problemas lineales, el vector $\bm{\vec{u}}$ puede ser resuelto de manera sencilla, mediante procedimientos básicos del álgebra lineal. Sin embargo, para problemas no lineales, la solución tiene que ser obtenida mediante una secuencia de pasos, en el cual cada uno de estos implica la modificación de la matriz de rigidez $\bm{[K]}$ y/o el vector global de carga $\bm{\vec{P}}$.

\subsection{Análisis dinámico}

En problemas de análisis dinámico, desplazamientos, velocidades, deformaciones, esfuerzos y cargas 
son dependientes del tiempo. El procedimiento de solución por elemento finito implica las siguientes 
consideraciones o pasos.\\

Primeramente, como en el caso estático, se debe idealizar el sólido en módelo de elementos finitos. 
Luego, se asume que los desplazamientos del elemento $e$ se pueden definir como:

\begin{equation}\label{eq:dynec1}
\vec{U}(x,y,z,t) = 
\left\{ \begin{matrix}
u(x,y,z,t) \\ v(x,y,z,t) \\ w(x,y,z,t) \\
\end{matrix} \right\} = 
[N(x,y,z)] \vec{Q}^{(e)}(t)
\end{equation}

Donde $\vec{U}$ es el vector de desplazamientos, $[N]$ es la matriz de funciones de forma, y $\vec{Q}_{(e)}$ 
es el vector de desplazamientos nodales que se considera como una función del tiempo.\\

Derivando las matrices de rigidez y masa características, además del vector de cargas, de la ecuación \ref{eq:dynec1} 
se pueden expresar las deformaciones como:

\begin{equation}
\vec{\varepsilon} = [B] \vec{Q}_^{(e)}
\end{equation}

y los esfuerzos como,

\begin{equation}
\vec{\sigma} = [D] \vec{\varepsilon} = [D][B] \vec{Q}^{(e)}
\end{equation}

Derivando la ecuación \ref{eq:dynec1} con respecto al tiempo, el campo de velocidades puede ser obtenido: 

\begin{equation}
\dot{\vec{U}}(x,y,z,t) = [N(x,y,z)]\dot{\vec{Q}}^{(e)} (t)
\end{equation}

donde $\dot{\vec{Q}}^{(e)}$ es el vector de velocidades nodales. Para derivar las ecuaciones de movimiento de 
una estructura, podemos usar las ecuaciones de Lagrange (\ref{eq:dynec1}) o el principio de Hamilton. Las 
ecuaciones de Lagrange están dadas por:

\begin{equation}
\frac{d}{dt} \left\{ \frac{\partial L}{\partial \dot{Q}} \right\}- 
\left\{ \frac{\partial L}{\partial Q} \right\} + 
\left\{ \frac{\partial R}{\partial \dot{Q}} \right\} = \{0\}
\end{equation}

donde,

\begin{equation}
L = T - \pi_p
\end{equation}

es llamada función Lagrangiana, T es la energía cinética, $\pi_p$ es la energía potencial, $R$ es 
la función de disipación, $Q$ es el desplazamiento nodal, y $\dot{Q}$ es la velocidad nodal.
La energía cinética y la energía potencial de un elemento $(e)$ puede ser expresada como:

\begin{equation}
T^{(e)}	= \frac{1}{2} \iiint_{V^{(e)}} \rho \dot{\vec{U}}^T \dot{\vec{U}} dV
\end{equation}

y,

\begin{equation}
\pi_p^{(e)} = \frac{1}{2} 
\iiint_{V^e} \vec{\varepsilon}^T \vec{\sigma} dV  - 
\iint_{S_1^e} \vec{U}^T \vec{\Phi} dS_1  - 
\iiint_{V^e} \vec{U}^T \vec{\Phi} dV
\end{equation}

donde $V^{(e)}$ es el volumen, $\rho$ es la densidad, $\dot{\vec{U}}$ es el vector de las velocidades del elemento $(e)$. 
Asumiendo la existencia de fuerzas disipativas proporcionales a las velocidades relativas, la función de disipación del 
elemento $(e)$ puede ser expresado como:

\begin{equation}
R^{(e)} = \frac{1}{2} \iiint_{V^{(e)}} \mu \dot{\vec{U}}^T \dot{\vec{U}} dV
\end{equation}

Donde $\mu$ es el coeficiente de amortiguamiento. Las expresiones para $T$, $\pi_p$ y $R$ pueden 
ser reescritas como:

\begin{equation}
T = \sum\limits_{e=1}^{E} T^{(e)} = \frac{1}{2} \dot{\vec{Q}}^T
\left[
\sum\limits_{e=1}^{E} \iiint\limits_{V^{(e)}} \rho [N]^T [N] dV
\right]
\dot{\vec{Q}}
\end{equation}

\begin{equation}
\pi_p = \sum\limits_{e=1}^{E} \pi_p^{(e)} = \frac{1}{2} \dot{\vec{Q}}^T
\left[
\sum\limits_{e=1}^{E} \iiint\limits_{V^{(e)}} [B]^T [D] [B] dV
\right] \vec{Q} - 
\vec{Q}^T 
\left(
\sum\limits_{e=1}^{E} \iint\limits_{S_1^{(e)}} [N]^T \vec{\Phi}(t) dS_1   + 
\iiint\limits_{V^{(e)}} [N]^T \vec{\Phi}(t) dV 
\right) -
\vec{Q}^T \vec{P}_c (t)
\end{equation}


\begin{equation}
R = \sum\limits_{e=1}^{E} R^{(e)} = \frac{1}{2} \dot{\vec{Q}}^T
\left[
\sum\limits_{e=1}^{E} \iiint\limits_{V^{(e)}} \mu [N]^T [N] dV 
\right]
\dot{\vec{Q}}
\end{equation}

Donde $\vec{Q}$ es el vector global de desplazamientos, $\dot{\vec{Q}}$ es el vector global de velocidades, 
y $\vec{P}_c$ es el vector de fuerzas concentradas nodales en la estructura del sólido. Las matrices que 
implican las integrales se pueden definir como:

\begin{equation}
[M^{(e)}] = \iiint\limits_{V^{(e)}} \rho [N]^T [N] dV
\end{equation}

\begin{equation}
[K^{(e)}] = \iiint\limits_{V^{(e)}} [B]^T [D] [B] dV
\end{equation}

\begin{equation}
[C^{(e)}] = \iiint\limits_{V^{(e)}} \mu [N]^T [N] dV
\end{equation}

\begin{equation}
[P_s^{(e)}] = \iint\limits_{S_1^{(e)}} [N]^T \vec{\Phi} dS_1
\end{equation}

\begin{equation}
[P_b^{(e)}] = \iint\limits_{V^{(e)}} [N]^T \vec{\Phi} dV
\end{equation}

Una vez se tienen las matrices características individuales, se deben  ensamblar las matrices globales y 
obtener las ecuaciones totales de movimiento.

\begin{equation}
T = \frac{1}{2} \dot{\vec{Q}}^T [M] \dot{\vec{Q}}
\end{equation}

\begin{equation}
\pi_p = \frac{1}{2} \dot{\vec{Q}}^T [K] \vec{Q} - \vec{Q}^T \vec{P}
\end{equation}

\begin{equation}
R = \frac{1}{2} \dot{\vec{Q}}^T [C] \dot{\vec{Q}}
\end{equation}

donde, 

$$
[M] = \sum\limits_{e=1}^{E} [M^{(e)}]
$$
$$[K] = \sum\limits_{e=1}^{E} [K^{(e)}]$$
$$[C] = \sum\limits_{e=1}^{E} [C^{(e)}]$$
$$
\vec{P}(t) = \sum\limits_{e=1}^{E} 
\left(
P_s^{(e)}(t) + P_b^{(e)}(t) 
\right)
+ P_c (t)
$$

Obtenemos la ecuación de movimiento del sólido:

\begin{equation}
[M] \ddot{\vec{Q}}(t) + [C] \dot{\vec{Q}}(t) + [K] \vec{Q} (t) = \vec{P} (t)
\end{equation}

Donde \ddot{\vec{Q}} es el vector de las aceleraciones nodales en el sistema global. Si se ignora 
las ecuación de movimiento puede ser escrita como:

\begin{equation}
[M] \ddot{\vec{Q}} + [K] \vec{Q} = \vec{P}
\end{equation}



\subsection{Solución implícita vs explícita}

De manera más general, el sistema de ecuaciones que describe un modelo de elemento finito en un determinado tiempo, se puede escribir como:

\begin{equation}
M\ddot{u} + C\dot{u} ̇+ f_{int} = f_{ext}
\end{equation}

Donde $C\dot{u}$ representa las fuerzas viscosas, mismas que deben incluirse cuando el sistema esté amortiguado artificialmente.\\ 
Existen, de manera general, dos tipos de métodos para resolver ecuaciones diferenciales  en problemas dependientes del tiempo: integración implícita e integración explícita. El método implícito de integración en el tiempo puede ser expresado como:

\begin{equation}
u_{n+1}=f(\dot{u}_{n+1},\ddot{u}_{n+1},u_n,\dot{u}_n,…)
\end{equation}

y el método explícito como:

\begin{equation}
u_{n+1}=f(u_n,\dot{u}_n,\ddot{u}_n,u_{n-1},\dot{u}_{n-1},…)
\end{equation}

El método implícito requiere conocer las derivadas temporales en el paso n+1, las cuales son desconocidas, mientras el método explícito está basado en los valores conocidos en el paso n. Si el problema es no lineal, el método implícito necesita un procedimiento iterativo para determinar los nuevos desplazamientos. Con el método explícito los nuevos desplazamientos se determinan directamente de los valores conocidos en pasos previos, evitando el uso de iteraciones adicionales.\\

La simulación de estampados se caracteriza por las diversas no linealidades presentes, debidas al material, y a los contactos entre los diversos cuerpos analizados. Sin embargo, usando la integración explícita estas no linealidades pueden ser tratadas sin mayores problemas. Por ello, en la mayoría de las simulaciones de estampados suelen ser conveniente el uso de algoritmos de integración explícita.

\section{Teoría de contactos}

\subsection{Algoritmo de contacto superficie-superficie}

El algoritmo de restricción que fue implementado está basado en el algoritmo desarrollado por 
Taylor y Flanagan [Revisar]. Este implica un enfoque simétrico de dos pasos con un parámetro 
de partición, $\beta$, que se establece entre la unidad positiva y negativa, donde $\beta=1$ 
y $\beta=-1$ corresponden a tratamiento de una forma con la superficie \textit{maestra} 
acumulando masa y fuerzas de la superficie \textit{esclava} (para $\beta = 1$) y viceversa 
(para $\beta = -1$).\\

En este enfoque de restricción las aceleraciones, velocidades y desplazamientos se actualizan 
primero a una configuración de prueba sin tener en cuenta las interacciones. Después de la 
actualización, una fuerza de penetración es calculada para el nodo esclavo como una función 
de la distancia de penetración $\Delta L$:

$$
mathbf{\f_p} = \frac{m_s \Delta L}{\Delta t^2}\mathbf{n},
$$

Donde $\mathbf{n}$ es el vector normal a la superficie maestra.\\

Se desea que la respuesta de la componente normal del vector de aceleración del nodo esclavo, 
$\mathbf{a_s}$, de un nodo esclavo residiendo en un segmento maestro $k$ sea consistente con 
el movimiento del segmento maestro en su porción de contacto $(s_c,t_c)$, por ejemplo:

$$
a_s = \phi_1 (s_c,t_c) a_{nk}^1 + \phi_2 (s_c,t_c) a_{nk}^2 + \phi_3 (s_c,t_c) a_{nk}^3 + \phi_4 (s_c,t_c) a_{nk}^4
$$

Para cada nodo esclavo en contacto y penetrando a través de la superficie maestra en 
su configuración de prueba, su masa nodal y fuerza de penetración dadas por la ecuación 
[REVISAR (Poner aquí referencia)] es acumulada a una masa y vector de fuerza global de 
la superficie maestra.

$$
\left(
m_k + \sum_s m_{ks}
\right)
\mathbf{a_{nk}}
=
\sum_s \mathbf{f_{ks}}
$$

donde:

$$
m_{ks} = \phi_k m_s
$$

$$
\mathbf{f_{ks}} = \phi{k} \mathbf{f_s}
$$

