\chapter*{ABSTRACT}
\addcontentsline{toc}{chapter}{ABSTRACT}

% \mybox{Abstract for thesis. That is, just a translation (ES-EN) from previous section}
The simulation by finite elements of forming processes is an auxiliary tool in the process of design and optimization of tooling. This thesis presents the analysis of the forming process of an AISI 1018 steel tube by means of a punch composed of two stages, namely: an U-bend and a tube closure. An explicit dynamic analysis was performed using the ANSYS/LS-DYNA\CR simulation software, taking into account all nonlinearities inherent to the actual process, such as the material model, geometries, dynamic nature of the phenomenon and interaction between the various components of the tooling and the workpiece. A maximum force of 45 tons was obtained, which is required for the last step of the process. The von Mises stresses obtained have a maximum of 84000 psi, which was used as a way of verifying that the analysis was within the plastic range of work. In order to evaluate the behavior of the workpiece during the work process, four models of material were used: multilinear, bilinear isotropic, bilinear kinematic and kinematic plasticity, obtaining very similar results for von Mises stresses variation in all cases, and with differences less than 20\% in the case of required forming force. A comparative analysis was made for several cases of selective mass scaling, in order to determine their influence on the results obtained. From the above it was estimated that by adding about 110\% of artificial mass it is possible to reduce the computation time by 90\%, with minimum variations of the von Mises stress and force obtained. An experimental analysis was also carried out using strain gauges to measure strains in a range up to 25 milistrains and the corresponding force to obtain them, comparing with that obtained by the simulation by finite elements, a similar behavior was observed and a maximum relative error of 9\%.