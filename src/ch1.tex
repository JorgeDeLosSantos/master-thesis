\chapter{Marco de referencia}

\section{Antecedentes}

Los procesos de estampado forman  parte importante de la industria metal-mecánica, muchas 
piezas metálicas se fabrican utilizando estos procesos, debido a los altos volúmenes de 
producción y a la rapidez de fabricación respecto a otros métodos como la fundición, forja 
o el mecanizado. Es empleado en gran variedad de sectores: electrodomésticos ,
automotriz, aeronáutico, naval, electrónico e informático y su objetivo es aprovechar al máximo 
el material para elaborar la mayor cantidad de piezas con el menor tiempo y costo posible. \\

El proceso de estampado, vital dentro del sector automotriz en México, genera una demanda 
de mercado con un valor de unos \$17,000 millones de dólares. Este monto es conformado por la 
suma de la proveeduría nacional, que aporta \$6,000 millones de dólares y las importaciones de 
piezas y componentes que utilizan este proceso de manufactura, valuadas en poco más de 
\$11,000 millones de dólares. Esto significa que la proveeduría nacional (fabricantes en el país) 
abastece el 35.2\% de la demanda que se presenta dentro de dicho proceso de la cadena de valor 
del sector automotriz. ~\cite{elhorizonte} \\

Las operaciones de troquelado y estampado pueden parecer actividades en las cuales no hay mayor 
ciencia o desarrollo de ingeniería, y muchas veces, los herramentales utilizados 
se fabrican mediante una aproximación dada por la experiencia de un jefe de taller o un 
mecánico matricero, y posteriormente se realizan pruebas para ajustar en donde sea necesario.
Está claro que para una industria pequeña esto puede funcionar, dado que la exigencia 
en la producción y las pérdidas económicas derivadas de un diseño inadecuado son mínimas. 
Pero, normalmente, en las industrias del sector automotriz los requerimientos en tiempo y 
costo dejan menos margen de maniobra e implican que los procesos de diseño sean desarrollados 
mediante una metodología que garantice la disponibilidad del producto en tiempo y forma.\\

En ese contexto, las herramientas computacionales, es decir paquetes de software CAD, CAE o CAM, 
proporcionan una ventaja competitiva considerable, puesto que permiten a las empresas agilizar 
los procesos de diseño y manufactura de los herramentales utilizados, que 
en conjunto con los procedimientos de producción y los aspectos de logística, posibilitan  
establecer una metodología ágil para el desarrollo de nuevos productos o de los ya existentes.


% Actualmente para garantizar el desarrollo a tiempo de un producto es necesario 

% La posibilidad de desarrollar simulaciones de los procesos de estampados metálicos fue 
% durante mucho tiempo casi una utopía para la industria. Los ingenieros 
% de procesos esperaban ser capaces de identificar posibles defectos en el formado en etapas 
% tempranas de diseño y/o desarrollo de los herramentales, y minimizar la necesidad de 
% modificaciones costosas de las herramientas en una serie de procesos de ensayo y error.\\


\section{Planteamiento del problema}

La empresa Bypasa S.A. de C.V. diseña y fabrica componentes utilizados en partes antivibratorias, 
para el sector automotriz, específicamente bujes y abrazaderas utilizadas en las suspensiones, 
partiendo desde la creación de los materiales poliméricos requeridos, la selección de la geometría 
y la validación del diseño. Sin embargo los herramentales: moldes y troqueles, que utiliza en sus 
procesos de producción son diseñados y fabricados por empresas extranjeras y/o nacionales. Por esto 
se está desarrollando un proyecto cuyo objetivo es poner en operación un centro de diseño, fabricación 
y validación de moldes y troqueles.\\

El modelado y simulación, por medio de métodos numéricos computacionales, de los procesos de fabricación 
involucrados en el desarrollo de los herramentales es una parte importante del proyecto, puesto que 
permitirá tomar decisiones respecto a los diseños en etapas tempranas, minimizando los costos derivados de 
los ajustes realizados después de ejecutar \textit{corridas} de prueba.\\

Se requiere simular el proceso de formado de un tubo de acero AISI 1018 que será fabricado con 
un herramental diseñado en el nuevo centro de desarrollo, con la finalidad de determinar 
la forma de la geometría final del producto, comparar con las especificaciones, determinar 
si es necesario un rediseño, además de obtener algunos datos como la fuerza de formado 
requerida para el proceso. 

\section{Justificación}

La simulación por elemento finito en el proceso de diseño de herramentales, así como en 
la simulación de procesos de estampado es una herramienta muy útil, puesto que representa un ahorro 
significativo de costos y tiempo, a la vez que permite establecer una metodología de trabajo más efectiva, minimizando las 
actividades de tipo prueba-error para la realización de ajustes.\\

Además de lo anterior, la simulación presenta la ventaja de poder variar parámetros que influyen en los procesos de 
estampado, de manera conveniente, sin que esto derive en gastos excesivos de recursos, con la finalidad de entender 
de mejor manera la influencia de ciertas propiedades o condiciones, e inclusive optimizar las características 
de un componente.



% \section{Hipótesis}

% Es posible simular el desarrollo del formado de un tubo para buje utilizando un software de simulación 
% por elemento finito y utilizar estos resultados como una herramienta auxiliar en el desarrollo de 
% nuevos herramentales.

\section{Objetivo general}

Simular y validar el proceso de formado de un tubo de acero AISI 1018.

\section{Objetivos particulares}
\begin{itemize}
\item Simular el desarrollo de formado del tubo.
\item Identificar fallas potenciales o características no deseables en el tubo.
\item Calcular de la fuerza requerida para completar el proceso de formado.
\item Validar los resultados de la simulación.
% \item Verificar las características geométricas y dimensionales del tubo.
%%% \item Caracterizar de forma experimental y por elemento finito el comportamiento elástico de la materia prima utilizada.
\end{itemize}


\section{Alcances}

\begin{itemize}
\item Desarrollo de un modelo de elemento finito del proceso de formado del tubo.
\item Simulación del modelo utilizando un análisis tipo dinámico-explícito.
\item Validación de la simulación.
% \item Publicación de un artículo.
\end{itemize}


\section{Estado del arte}

La posibilidad de desarrollar simulaciones de los procesos de estampados metálicos fue durante mucho tiempo 
un deseo inalcanzable para la industria de estampados. Los ingenieros de procesos esperaban ser capaces de 
identificar posibles defectos en el formado en etapas tempranas de diseño y/o desarrollo de los herramentales, 
y minimizar la necesidad de modificaciones costosas de las herramientas en una serie de procesos de ensayo y error. \\

El modelado de problemas de estampado de partes metálicas requiere una precisión considerable en la caracterización 
de efectos como el comportamiento no lineal de un material, grandes deformaciones y condiciones de contacto entre la
herramienta y la parte a estampar.que derivan  en algoritmos complejos.\cite{banabic2000}\\

La primera formulación teóricamente correcta de problemas de formado de metales fue presentada por 
Wang y Budiansky ~\cite{wang1978} en 1978. El método presentado fue una formulación total lagrangiana 
e involucraba elementos triangulares membrana de deformación constante. La solución implementada fue 
un esquema incremental Euleriano hacia adelante. Los métodos basados en un 
esquema de solución como el anterior son llamados como métodos estáticos-explícitos.\\

En el inicio de la década de los 90's hubo un incremento considerable en la utilización de la simulación de estampado 
metálico dentro de la industria y a mediados de esta década la mayoría de las compañías en la industria automotriz 
establecieron las simulaciones de estampado como aspectos elementales en el desarrollo de sus procesos. 
Actualmente existen programas de computadora altamente especializados en la simulación de estampados, siendo AutoForm 
uno de los más utilizados, este surgió como un proyecto de investigación en el ETH de Zurich en los inicios de los 90's. 
El código está basado en un enfoque estático-implícito, pero utiliza algunos algoritmos innovativos que le permiten 
una estabilidad y eficiencia computacional competitiva respecto a los códigos de tipo dinámico-explícito.  ~\cite{banabic2000}\\

Para la simulación numérica de procesos de formado de tubo mediante doblados sucesivos en UO, 
similares al proceso a desarrollar, se tienen algunos trabajos anteriores, mismos que se describen 
a continuación.\\

Huang & Leu ~\cite{huang1995} desarrollaron un código de análisis elasto-plástico por elemento finito, 
basado en una formulación lagrangiana modificada, para simular proceso de doblado UO en placas metálicas, 
bajo condiciones de deformación plana. Para realizar el análisis del proceso completo, dividieron este 
en tres pasos de carga o configuraciones que se muestran en la figura ~\ref{fig:pasos_formado_01}, 
doblado en U, descarga y doblado final en O. Aplicaron además simetría debido a la disposicion de las 
herramientas y el blank a formar, simplificando aún más el análisis. Utilizaron un coeficiente 
de fricción de $\mu = 0.02$ y el espesor de la chapa fue de 6 mm. \\
% En la figura ~\ref{fig:shape_sequence} 
% se puede observar la secuencia del desarrollo de la geometría obtenida mediante el análisis realizado. \\

\begin{center}
\includegraphics[scale=1.0]{src/ch1/uo-bending.png}
\captionof{figure}{Proceso de doblado UO, a) Doblado en U b) Descarga c) Doblado en O \cite{huang1995}}
 \label{fig:pasos_formado_01}
\end{center}

% \begin{center} \label{fig:shape_sequence}
% \includegraphics[scale=0.75]{src/ch1/shape_sequence.png}
% \captionof{figure}{Secuencia obtenida mediante el proceso de doblado UO. \cite{huang1995}}
% \end{center}

Chen y Huang ~\cite{chen2007} de manera similar a ~\cite{huang1995} simularon un proceso de doblado 
UO, con las mismas etapas: doblado en U, descarga y doblado en O. El espesor de lámina de la pieza 
de trabajo fue de 6 mm y un ancho de 10 mm. Utilizaron una ecuación exponencial para la relación 
esfuerzo-deformación.  En las figuras ~\ref{fig:u_bend} y ~\ref{fig:o_bend} se muestran el modelo 
FEM de los pasos de formado correspondientes. Los resultados obtenidos fueron la distribución 
de esfuerzos de Von Mises y la forma geométrica resultante, medidas para ciertos intervalos 
de desplazamiento del punzon formador superior.

\begin{center}
\includegraphics[scale=0.65]{src/ch1/u_bend.png}
\captionof{figure}{Modelo de elementos finitos del doblado en U \cite{chen2007}}
\label{fig:u_bend}
\end{center}

\begin{center}
\includegraphics[scale=0.65]{src/ch1/o_bend.png}
\captionof{figure}{Modelo de elementos finitos del doblado en O \cite{chen2007}}
\label{fig:o_bend}
\end{center}