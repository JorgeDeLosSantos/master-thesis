\chapter*{NOMENCLATURA}
\addcontentsline{toc}{chapter}{NOMENCLATURA}

% \nomenclature{$c$}{Speed of light in a vacuum inertial frame}
% \nomenclature{$h$}{Planck constant} 

% \printnomenclature

% Simple nomeclature, nomencl package don't work

\begin{table}[h]
\def\arraystretch{1.15}
\begin{tabular}{p{4cm} p{12cm}}

CAE      &                       		      Ingeniería asistida por computadora \\
CAD &                                  		  Diseño asistido por computadora \\
CAM & 										  Manufactura asisitida por computadora \\
FEA &                                  		  Análisis por elementos finitos \\
FEM & 							              Método de los elementos finitos \\
AISI &                                        Instituto Americano del Hierro y el Acero \\
SAE &                                         Sociedad de Ingenieros Automotrices \\
ASTM &                                        Sociedad Americana para Pruebas y Materiales \\
$\sigma_1, \sigma_2, \sigma_3$ &              Esfuerzos principales \\
$\sigma_x, \sigma_y, \sigma_z$ &			  Esfuerzo en la dirección X,Y,Z \\
$\sigma_{nom}$ & 							  Esfuerzo nominal \\
$\sigma_{t}$ & 								  Esfuerzo verdadero \\
$\varepsilon_1, \varepsilon_2, \varepsilon_3$ & Deformaciones principales \\
$\varepsilon_x, \varepsilon_y, \varepsilon_z$ & Deformación en la dirección X,Y,Z \\
$\varepsilon_{nom} $ &						  Deformación nominal \\
$\varepsilon_{t}$ &  						  Deformación verdadera \\
$\varepsilon_{pl}$ & 						  Deformación plástica \\
$ S_y $ &                                     Esfuerzo de fluencia \\
$ k_R $ &                                     Factor de springback \\
$ \alpha_i $ & 								  Ángulo de doblado inicial \\
$ \alpha_f $ & 								  Ángulo de doblado final \\
$ r_i $ & 									  Radio inicial interno de doblado \\
$ r_f $ & 									  Radio final interno de doblado \\
$ t $ &                                       Espesor de lámina metálica \\
% $ K $ &                                       Matriz global de rigidez \\
% $ M $ & 									  Matriz de masas \\
% $ \vec{u} $ &                                 Vector de desplazamientos \\
% $ \vec{P} $ &                                 Vector de fuerzas nodales \\
{\tt RBUX, RBUY, RBUZ} &                      Desplazamiento de cuerpo rígido en dirección X,Y,Z \\
{\tt UX, UY, UZ} &                            Desplazamiento en dirección X,Y,Z \\
{\tt ROTX, ROTY, ROTZ} & 					  Rotación en dirección X,Y,Z \\
in &                                          Pulgada \\
lbf &                                         Libra-fuerza \\
psi &                                         Libra-fuerza por pulgada cuadrada (lbf/in$^2$) \\

\end{tabular}
\end{table}