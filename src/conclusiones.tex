\chapter*{Conclusiones}
\addcontentsline{toc}{chapter}{Conclusiones}

% \mybox{CONCLUSIONES DEL TRABAJO DE TESIS}

La simulación por elementos finitos es una herramienta \\

Es importante identificar cuándo se tienen análisis que pueden tratarse como casos simplificados 
de esfuerzo plano o deformación plana, ya que esto puede representar una reducción significativa 
de tiempo y recursos computacionales. En el caso del modelo analizado en este proyecto, asumir 
una condición de deformación plana ha generado resultados bastante similares a los obtenidos 
mediante la simulación del modelo tridimensional. Cuando se hace un análisis de tipo 
deformación plana es importante tomar en cuenta que este se realiza considerando un especimen 
de espesor unitario, y por tanto hay que hacer los ajustes o correcciones necesarias en 
caso de que el modelo completo difiera de esta característica. \\

El fenómeno de la fricción en un proceso de doblado no es tan significativo, puesto que no 
es necesario realizar una lubricación adicional o especial, como puede ocurrir en casos 
de embutición profunda, de modo que en análisis por elementos finitos se puede asumir 
un coeficiente friccional ordinario de interacción metal-metal de 0.2, el cual da resultados 
aceptables. De hecho, variando el coeficiente de fricción en el análisis realizado, se observó 
muy poca diferencia en las geometrías deformadas y los resultados obtenidos en un rango de 0.05-0.3, 
con valores menores o casi nulos de fricción la pieza tiende a deslizarse sobre el formador 
inferior y en consecuencia se obtiene una geometría resultante asimétrica.\\

El escalamiento de masa selectivo en un análisis de tipo dinámico explícito representa una 
opción bastante viable para cuando se tienen recursos computacionales limitados y se 
requiere ejecutar un análisis que no admite otro tipo de simplificaciones. Esto requiere 
obviamente de un criterio de selección del tamaño de los pasos de tiempo y/o del porcentaje 
de \textit{masa artificial} que se adiciona. De acuerdo al manual teórico del software 
utilizado, es posible, en la mayoría de los casos, disminuir en un 50\% el tiempo de CPU 
empleado agregando una cantidad alrededor de 0.001\% de masa adicional.\\