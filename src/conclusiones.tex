\chapter*{CONCLUSIONES}
\addcontentsline{toc}{chapter}{CONCLUSIONES}

% \mybox{CONCLUSIONES DEL TRABAJO DE TESIS}

En este proyecto se desarrolló la simulación por elementos finitos del proceso de formado 
de un tubo de acero AISI 1018, mediante un enfoque dinámico explícito. De acuerdo a lo 
planteado inicialmente se obtuvieron resultados aceptables, en términos de las geometrías 
resultantes y las fuerzas requeridas respecto a la capacidad de trabajo de las prensas 
disponibles en producción.\\

La simulación por elementos finitos es una herramienta auxiliar muy útil en el proceso de 
diseño y validación de herramentales y proporciona una ventaja competitiva si se utiliza 
de forma adecuada, para lo cual hace falta caracterizar de manera correcta el fenónemo 
que se está simulando, tomando en cuenta todas las cuestiones que puedan influir 
en el proceso. Del análisis por elementos finitos realizado durante este proyecto 
se pudieron extraer algunas consideraciones y/o conclusiones que resultan válidas 
para la mayoría de los procesos de doblado mediante un análisis de tipo dinámico explícito, mismas 
que se describen enseguida.\\

Es importante identificar cuándo se tienen análisis que pueden tratarse como casos simplificados 
de esfuerzo plano o deformación plana, ya que esto puede representar una reducción significativa 
de tiempo y recursos computacionales. En el caso del modelo analizado en este proyecto, asumir 
una condición de deformación plana ha generado resultados bastante similares a los obtenidos 
mediante la simulación del modelo tridimensional, con un error relativo de 4\% para la fuerza 
de formado máxima en el cerrado del tubo, y menor al 1\% en la etapa del doblado en U. 
Respecto al tiempo de cómputo requerido, se pudo observar que un análisis bidimensional 
permite reducirlo hasta en un 97\% respecto a un modelo tridimensional.
Cuando se hace un análisis de tipo deformación plana es importante tomar en cuenta que este 
se realiza considerando un especimen de espesor unitario, y por tanto hay que hacer los ajustes 
o correcciones necesarias en caso de que el modelo completo difiera de esta característica. \\

El fenómeno de la fricción en un proceso de doblado no es tan significativo, puesto que no 
es necesario realizar una lubricación adicional o especial, como puede ocurrir en casos 
de embutición profunda, de modo que en análisis por elementos finitos se puede asumir 
un coeficiente friccional ordinario de interacción metal-metal de 0.2, el cual da resultados 
aceptables. De hecho, variando el coeficiente de fricción en el análisis realizado, se observó 
muy poca diferencia en las geometrías deformadas y los resultados obtenidos en un rango de 0.05-0.3, 
con valores menores o casi nulos de fricción la pieza tiende a deslizarse sobre el formador 
inferior y en consecuencia se obtiene una geometría resultante asimétrica.\\

El escalamiento de masa selectivo en un análisis de tipo dinámico explícito representa una 
opción bastante viable para cuando se tienen recursos computacionales limitados y se 
requiere ejecutar un análisis que no admite otro tipo de simplificaciones. Esto requiere 
obviamente de un criterio de selección del tamaño de los pasos de tiempo y/o del porcentaje 
de \textit{masa artificial} que se adiciona. De acuerdo al manual teórico del software 
utilizado, es posible, en la mayoría de los casos, disminuir en un 50\% el tiempo de CPU 
empleado agregando una cantidad alrededor de 0.001\% de masa adicional.\\

Para el análisis experimental se utilizaron galgas extensométricas, cuya lectura se redujo  
al doblado en U del primer paso, debido a las limitaciones inherentes a la instrumentación y las 
grandes deformaciones presentes en el proceso de formado. Comparando los resultados obtenidos 
de forma experimental con los de la simulación del modelo bidimensional, se observó un comportamiento 
bastante similar en la curva de la relación fuerza-deformación, con una correlación del 98\% , 
lo cual implica que el modelo del material utilizado y las propiedades establecidas en el análisis 
por elementos finitos proporcionan resultados adecuados para procesos de este tipo.  \\

Con lo realizado en este proyecto, la empresa tendrá un precedente y punto de partida para la 
implementación del uso de software de análisis por elementos finitos para la simulación de 
partes estampadas, desde el tipo de análisis a realizar sabiendo a priori las consideraciones aquí 
expuestas, los parámetros de interés para la simulación, modelos constitutivos, así como 
conocer los datos de salida que pueden obtenerse de una simulación de este tipo. Puesto que la 
mayoría de las partes estampadas fabricadas en la empresa, se realizan mediante operaciones de 
doblado con láminas de similar espesor, lo aquí estudiado es aplicable y adaptable para 
la simulación de desarrollos futuros.