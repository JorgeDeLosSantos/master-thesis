\documentclass[12pt,letterpaper]{book}
\usepackage[utf8]{inputenc}
\usepackage[spanish,es-tabla]{babel}
\decimalpoint
\let\cleardoublepage\clearpage
\usepackage{amsmath}
\usepackage{amsfonts}
\usepackage{amssymb}
\usepackage{esint}
\usepackage{color}
\usepackage{graphicx}
\usepackage{anysize}
\usepackage{anyfontsize}
\usepackage{pdfpages}
\usepackage{makeidx}
\makeindex 
\usepackage{epstopdf}
\usepackage[x11names,table]{xcolor}
\usepackage{tikz}
\usepackage{tcolorbox}
\usepackage[hidelinks]{hyperref}
\usepackage{caption}
\usepackage{listings}
\usepackage{bm}
% Margenes
\usepackage[left=3cm,top=2.5cm,right=2.5cm,bottom=2.5cm]{geometry}

\setlength{\parindent}{0cm}
\tcbset{colback=green!5!white, colframe=gray!10!black, coltitle=green!20!black, 
fonttitle=\bfseries, colbacktitle=white, coltext=gray!30!black}
\addto\captionsspanish{
    \renewcommand{\figurename}{{\bf Figura}}% 
}
\usepackage{epigraph}
\usepackage{xcolor}
\usepackage{textcomp}
\usepackage{gensymb}

% Colores
\definecolor{verdep}{rgb}{0.5,0.5,0.9}
\definecolor{ccap}{rgb}{0.2,0.2,0.2}
\definecolor{csec}{rgb}{0.4,0.4,0.4}
\definecolor{csubsec}{rgb}{0.6,0.6,0.6}
\definecolor{cenun}{rgb}{0.2,0.2,0.3}
\definecolor{csol}{rgb}{0.2,0.8,0.1}
\definecolor{backcode}{rgb}{0.95,0.95,0.99}
\definecolor{dkgreen}{rgb}{0,0.6,0}
\definecolor{gray}{rgb}{0.5,0.5,0.5}
\definecolor{mauve}{rgb}{0.58,0,0.82}

% Nuevos comandos

\usepackage{titlesec}%--
\newcommand{\hsp}{\hspace{5pt}}
%\titleformat{\chapter}[hang]{\huge\bfseries\color{ccap}}
%{\color{verdep}{\vrule height 2.5cm width 1mm}\hsp{\fontsize{100}{5}\selectfont\thechapter}\hsp%
%{\vrule height 2.5cm width 1mm}\hsp{\fontsize{30}{5}\selectfont}}{5pt}{\huge\bfseries}

% \titleformat{\chapter}[hang]{\huge\bfseries\color{ccap}}
% {\color{verdep}\hsp{\fontsize{30}{5}\selectfont Capítulo \thechapter.\\}\hsp%
% \hsp{\fontsize{30}{5}\selectfont}}{5pt}{\huge\bfseries}

% \titleformat{\section}[hang]{\normalfont\color{csec}}%
% {\filright\large\enspace\thesection\enspace}%
% {8pt}{\Large\bfseries\filright}%

% \titleformat{\subsection}[hang]{\normalfont\color{csec}}%
% {\filright\large\enspace\thesubsection\enspace}%
% {8pt}{\large\bfseries\filright}%

% Code

\lstnewenvironment{apdl}{\lstset{frame=single,
    frameround=tttt,
    backgroundcolor=\color{backcode},
    language={},
    aboveskip=3mm,
    belowskip=3mm,
    showstringspaces=false,
    columns=flexible,
    basicstyle={\small\ttfamily},
    numbers=none,
    numberstyle=\tiny\color{gray},
    keywordstyle=\color{blue},
    commentstyle=\color{dkgreen},
    stringstyle=\color{mauve},
    breaklines=true,
    breakatwhitespace=true,
    tabsize=3,
    extendedchars=true,
    inputencoding=utf8,
    literate=%
    {°}{{\,\,$^\circ$\,\,}}1
    {á}{{\'a}}1
    {é}{{\'e}}1
    {í}{{\'i}}1
    {ó}{{\'o}}1
    {ú}{{\'u}}1
    {Á}{{\'A}}1
    {É}{{\'E}}1
    {Í}{{\'I}}1
    {Ó}{{\'O}}1
    {Ú}{{\'U}}1
}}{}



\author{Pedro Jorge De Los Santos Lara}
\title{Simulación por elemento finito del proceso de estampado de un tubo de acero SAE 1018}


% ======================================================================================================
\begin{document}
\maketitle
\tableofcontents


\chapter{Marco de referencia}
\section{Antecedentes}
\section{Planteamiento del problema}
\section{Estado del arte}
\section{Justificación}
\section{Objetivo general}
\section{Objetivos particulares}
\section{Alcances}


\chapter{Marco teórico}
\section{Procesos de formado}
\section{Teoría de plasticidad}
\section{El método de los elementos finitos}


\chapter{Metodología}
\section{Análisis por elemento finito}
\section{Análisis experimental}


\chapter{Resultados}
\section{Fuerza de formado requerida}
\section{Esfuerzos y deformaciones}



\addcontentsline{toc}{chapter}{Conclusiones}
\chapter*{Conclusiones}

\addcontentsline{toc}{chapter}{Referencias}
\chapter*{Referencias}

\addcontentsline{toc}{chapter}{Anexos}
\chapter*{Anexos}


%\newpage
% \chapter*{Fuentes de información}
\addcontentsline{toc}{chapter}{Fuentes de información}

\begin{enumerate}
\item Banabic, Dorel (2010). \textit{Sheet metal forming processes, constitutive modelling and numerical simulation.} Rumanía: Springer.

\item Wang N-M, Budiansky B (1978). \textit{Analysis of sheet metal stamping by finite element method.} Journal of Applied Mechanics, Transaction ASME 45:73–82

\item Huang, Y. & Leu, D. (1995). \textit{An elasto-plastic finite-element simulation of successive UO-bending processes of sheet metal.} Department of Mechanical Engineering, National Taiwan Institute of Technology. Taiwan.

\item Herynk, M. et. al. (2007). \textit{Effects of the UOE/UOC pipe manufacturing processes on pipe collapse pressure.} Research Center for Mechanics of Solids, Structures and Materials, The University of Texas at Austin. USA.

\item Nielsen, K.B. (1997). \textit{Sheet metal forming simulation using explicit finite element methods}. Department of production, Aalborg University. Dinamarca.

\item Hosford, W. (2005). \textit{Mechanical behavior of materials}. New York: Cambridge University Press.

%\item Arwidson, Claes (2005). \textit{Numerical simulation of sheet metal forming for high strenght steels (Tesis de grado)}. Lulea University of Technology. Lulea, Suecia.
%\item Bahamonde N, Guaranda W (2007). \textit{Simulación del proceso de estampado en chapas metálicas y su recuperación elástica a través del software de elementos finitos Stampack (Tesis de grado)}. Escuela Politécnica Nacional. Quito, Ecuador.
%\item Dutton, Trevor (2004). \textit{Review of sheet metal forming simulation progress to date, future developments}. 8th International LS-DYNA users conference.
%\item Guzmán, Javier (2010). \textit{Rediseño de un herramental de embutido y su implementación para la ejecución del ensayo Erichsen (Tesis de grado)}. Universidad Nacional Autónoma de México. D.F., México.
%\item Honecker A, Mattiasson K (1989). \textit{Finite element procedures for 3D sheet forming simulation.} In: Thompson EG, Wood RD, Zienkiewicz OC, Samuelsson A (eds) NUMIFORM’89, AABalkema, Fort Collins
%\item Lindberg, Filip (2012). \textit{Sheet metal forming simulations with FEM (Tesis de maestría)}. Umea University. Umea, Suecia.

% SU10206715669304025 % Caliente Sport

\end{enumerate}

% \bibliographystyle{ieeetr}
% \bibliography{bibs}

\end{document}

% chrome://settings/resetProfileSettings